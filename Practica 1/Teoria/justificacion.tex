\documentclass[12pt]{article}

\usepackage[spanish]{babel}
\usepackage{amsmath}
\usepackage{graphicx}
\usepackage[dvipsnames]{xcolor}
\usepackage{amsthm}
\usepackage{amsfonts}
\usepackage{amssymb}
\usepackage[a4paper, total={6.5in, 10in}]{geometry}
\usepackage[framemethod=TikZ]{mdframed}
\usepackage{bm}
\usepackage{mathrsfs}
\usepackage{dsfont}
\usepackage{fancyhdr}

\begin{document}


\subsection*{Cambio a la base de Taylor}
\noindent
Vamos a justificar el cambio de base del polinomio de Hermite a la base de Taylor en $x_0$. Calculamos las diferencias divididas para expresar el polinomio de Hermite como
\begin{multline*}
	H(x) = f[x_0] + f[x_0, x_0] (x - x_0) + f[x_0, x_0, x_0] ( x-  x_0)^2 + f[x_0, x_0, x_0, x_1] (x - x_0)^3 + \\ + f[x_0, x_0, x_0, x_1, x_1] (x - x_0)^3 (x - x_1) + f[x_0, x_0, x_0, x_1, x_1, x_1] (x - x_0)^3 (x - x_1)^2
\end{multline*}
que por simplicidad escribimos
$$H(x) = a_0 + a_1 (x - x_0) + a_2 ( x-  x_0)^2 + a_3 (x - x_0)^3 + a_4 (x - x_0)^3 (x - x_1) + a_5 (x - x_0)^3 (x - x_1)^2$$
y queremos expresarlo en términos de $(x -x_0)^i$. Los cuatro primeros términos ya están expresados en esos términos luego solo nos queda expresar los dos últimos:

\begin{align*}
	[1]\;\; a_4 (x - x_0)^3 (x - x_1) & = a_4 (x - x_0)^3 (x - x_0 + x_0 - x_1) = a_4 (x - x_0)^3 (x - x_0) + a_4 (x - x_0)^3 (x_0 - x_1) \\
	&= a_4 (x_0 - x_1) (x - x_0)^3 + a_4 (x-x_0)^4 
\end{align*}
\noindent
\begin{align*}
	[2]\;\; a_5 (x - x_0)^3 (x - x_1)^2 &= a_5 (x-x_0)^3 (x - x_0 + x_0 - x_1)^2  = a_5 (x-x_0)^3 \left[(x - x_0) + (x_0 - x_1)\right]^2  \\ &= a_5 (x-x_0)^3 \left[(x - x_0)^2 + 2(x - x_0) (x_0 - x_1) + (x_0 - x_1)^2\right] \\
	&= a_5 (x-x_0)^5 + 2 a_5 (x_0 - x_1) (x - x_0)^4 + a_5 (x_0 - x_1)^2 (x-x_0)^3
\end{align*}
\noindent
Si suponemos 
$$H(x) = d_0 + d_1 (x - x_0) + d_2 (x - x_0)^2 + d_3 (x - x_0)^3 + d_4 (x - x_4)^4 + d_5 (x - x_0)^5$$
entonces si agrupamos terminos en $H(x), [1]$ y $[2]$ los coeficientes resultan ser
\begin{align*}
	d_0 &= a_0 & d_3 &= a_3 + a_4 (x_0 - x_1) + a_5(x_0 - x_1)^2 \\
	d_1 &= a_1 & d_4 &= a_4 + 2 a_5 (x_0 - x_1) \\ 
	d_2 &= a_2 & d_5 &= a_5
\end{align*}
\qed

\end{document}
