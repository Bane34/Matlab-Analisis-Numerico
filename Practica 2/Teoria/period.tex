\documentclass[12pt]{article}

\usepackage[spanish]{babel}
\usepackage{amsmath}
\usepackage{graphicx}
\usepackage[dvipsnames]{xcolor}
\usepackage{amsthm}
\usepackage{amsfonts}
\usepackage{amssymb}
\usepackage[a4paper, total={6.5in, 10in}]{geometry}
\usepackage[framemethod=TikZ]{mdframed}
\usepackage{bm}
\usepackage{mathrsfs}
\usepackage{dsfont}
\usepackage{fancyhdr}
\usepackage{bm}
\usepackage{enumitem}
\usepackage{dsfont}

\usepackage{mdframed}
\usepackage{lipsum}

\newmdtheoremenv{lema}{Lema}

\newenvironment{demostracion}{\noindent\textit{Demostración: } $ $ \newline }{\newline\qed}

\newcommand{\N}{\mathbb{N}}
\newcommand{\R}{\mathbb{R}}
\newcommand{\RR}{\rm I\!R}
\newcommand{\NN}{{\rm I\!N}}
\newcommand{\CC}{$\mathds{C}$}
\newcommand{\QQ}{${\mathds{Q}}$}
\newcommand{\KK}{{\rm I\!K}}
\newcommand{\ZZ}{$\mathds{Z}$}
\newcommand{\abs}[1]{\left|#1\right|}

\title{\textbf{Splines cúbicos periódicos}}
\date{}
\begin{document}
\maketitle
\noindent
\begin{mdframed}
	Sea $x_0 < x_1 < \dots < x_n$ una partición del intervalo $[x_0, x_n]$ $(n\geq 3)$, y valores de una función en todos los nodos de dicha partición salvo en el último $f_0, f_1,\dots,f_{n-1}$. Queremos construir un spline cúbico que interpole dichos datos con la condición de periodicidad, \textit{i.e.}, si $S(x)$ es el spline que buscamos, 
	$$S(x_0) = S(x_n) = f_0 \quad S'(x_0) = S'(x_n) \quad S''(x_0) = S''(x_n)$$
\end{mdframed}

Supongamos 
$$S(x) = \begin{cases}
	s_1(x) & \textrm{ si } x\in [x_0, x_1) \\ 
	s_2(x) & \textrm{ si } x\in [x_1, x_2) \\ 
	\vdots & \vdots \\ 
	s_n(x) & \textrm{ si } x \in [x_{n-1}, x_n]
\end{cases}$$

Para cada $i =  1, \dots, n$ calculamos los coeficientes del polinomio cúbico $s_i$ mediante la correspondiente tabla de diferencias dividas en el intervalo $[x_{i-1}, x_i]$:

\begin{center}
	\begin{tabular}{c | c c c c}
		$x$ & $f[x_{i-1}^*]$ & $f[x_{i-1}^*, x_i^*]$ & $f[x_{i-1}^*, x_i^*, x_{i + 1}^*]$ & $f[x_{i-1}^*, x_i^*, x_{i+1}^*, x_{i+2}^*]$ \\
		\hline
		$x_{i-1}$ & $f_{i-1}$ & 0 & 0 & 0 \\ 
		$x_{i-1}$ & $f_{i-1}$ & $H_{i-1}'$ & 0 & 0 \\ 
		$x_i$ & $f_i$ & $\frac{f_i - f_{i-1}}{x_i - x_{i-1}}$ & $\frac{H[x_{i-1}, x_i] - H_{i-1}'}{(x_i - x_{i-1})}$ & 0 \\ 
		$x_i$ & $f_i$ & $H_i'$ & $\frac{  H_{i}' - H[x_{i-1}, x_i]}{(x_i - x_{i-1})}$ & $\frac{H_i' - 2 H [x_{i-1}, x_i] + H_{i-1}'}{(x_i - x_{i-1})^2} $
	\end{tabular}
\end{center}
con $H[x_{i-1}, x_i] = \frac{f_i - f_{i-1}}{x_i - x_{i-1}}$. Así, la expresión de $s_i$ resulta en

$$s_i(x) = f_{i-1} + H_{i-1}'(x-x_{i-1}) + \frac{H[x_{i-1}, x_i] - H_{i-1}'}{(x_i - x_{i-1})} (x-x_{i-1})^2 + \frac{H_i' - 2 H [x_{i-1}, x_i] + H_{i-1}'}{(x_i - x_{i-1})^2} (x - x_{i-1})^2 (x - x_i) $$

Con esta construcción ya tenemos asegurada la continuidad y la continuidad $\mathscr{C}^1$ en los nodos de la partición. Falta imponer la condición de continuidad $\mathscr{C}^2$ y las condiciones de periodicidad. Para ello vamos a expresar cada $s_i(x)$ con $i=1,\dots, n$ en la base de Taylor con respecto al nodo $x_{i-1}$. Notamos que 

\begin{multline*}
	\frac{H_i' - 2 H [x_{i-1}, x_i] + H_{i-1}'}{(x_i - x_{i-1})^2} (x - x_{i-1})^2 (x - x_i) \\ = \frac{H_i' - 2 H [x_{i-1}, x_i] + H_{i-1}'}{(x_i - x_{i-1})^2} (x - x_{i-1})^2 (x - x_{i-1} + x_{i-1} - x_i) 
	\\ = \frac{H_i' - 2 H [x_{i-1}, x_i] + H_{i-1}'}{(x_i - x_{i-1})^2} (x - x_{i-1})^3 - \frac{H_i' - 2 H [x_{i-1}, x_i] + H_{i-1}'}{(x_i - x_{i-1})} (x - x_{i-1})^2
\end{multline*}
y agrupando términos
\begin{align*}
	s_i(x) = f_{i-1} &+ H_{i-1}' (x - x_i) + \frac{3 H [x_{i-1}, x_i] -2 H_{i-1}' - H_i'}{x_i - x_{i-1}} ( x-x_{i-1})^2 \\ &+ \frac{H_{i-1}' - 2 H[x_{i-1}, x_i] + H_i'}{(x_i - x_{i-1})^2} (x - x_{i-1})^3
\end{align*}
Derivando dos veces obtenemos para cada $i = 1,\dots, n$
$$s_i''(x) = 2 \frac{3 H [x_{i-1}, x_i] -2 H_{i-1}' - H_i'}{x_i - x_{i-1}} + 6\frac{H_{i-1}' - 2 H[x_{i-1}, x_i] + H_i'}{(x_i - x_{i-1})^2} (x - x_{i-1}) $$


Ahora imponemos la condición de continuidad $\mathscr{C}^2$ en los nodos $x_i$ con $ i = 1, \dots, n - 1$:
\begin{align*}
	s_i''(x_i) &= \frac{4 H_i' + 2 H_{i - 1}' - 6 H[x_{i-1}, x_i]}{x_i - x_{i-1}} \\
	s_{i+1}''(x_i) &= \frac{6 H[x_i, x_{i+1}] - 4 H_i' - 2H_{i+1}'}{x_{i+1}- x_i}
\end{align*}
e igualando se obtiene que 
$$\frac{H_{i-1}'}{h_i} + \left(\frac{2}{h_{i}} + \frac{2}{h_{i+1}}\right) H_i' + \frac{H_{i+1}'}{h_{i+1}} = 3 \left[\frac{H[x_{i-1} + x_i]}{h_i} + \frac{H[x_i, x_{i+1}]}{h_{i+1}}\right] \quad i = 1, 2, \dots, n - 1$$

En cuanto a las condiciones de frontera, por construcción se cumple que $s_1(x_0) = s_n(x_n)$ y $s_1''(x_0) = s_n'(x_n)$, falta ver la continuidad $\mathscr{C}^2$. Haciendo $s_1''(x_0) = s_n''(x_n)$ se tiene que 
$$\frac{H_0'}{h_1} + \left(\frac{2}{h_0} + \frac{2}{h_1}\right) H_1' + \frac{H_{n-1}'}{h_{n}} = 3\left[\frac{H[x_0, x_1]}{h_1} + \frac{H[x_{n-1}, x_n]}{h_n}\right]$$ 
en forma matricial el sistema es, poniendo $d_i = H[x_{i-1}, x_i]$, 

$$\begin{bmatrix}
	\frac{2}{h_1} + \frac{2}{h_n} & \frac{1}{h_1} & 0 & \dots & \dots & 0 & \frac{1}{h_n} \\
	\frac{1}{h_1} & \frac{2}{h_1} + \frac{2}{h_2} & \frac{1}{h_2} & \dots & \dots & 0 & 0 \\
	\vdots & \ddots & \ddots & \ddots & \vdots & \vdots & \vdots \\
	0 & \dots & \frac{1}{h_i} & \frac{2}{h_i} + \frac{2}{h_{i+1}} & \frac{1}{h_{i+1}} & \dots & 0 \\
	\vdots & \vdots & \vdots & \ddots & \ddots & \ddots & \vdots \\
	0 & 0 & \dots & \dots & \frac{1}{h_{n-2}} &  \frac{2}{h_{n-1}} + \frac{2}{h_{n-1}} & \frac{1}{h_{n-1}} \\
	
	\frac{1}{h_n} & 0 & \dots & \dots & 0 &  \frac{1}{h_{n-1}} & \frac{2}{h_{n-1}} + \frac{2}{h_n}
	
\end{bmatrix} \begin{bmatrix}

H_0' \\ H_1' \\ \vdots  \\ H_i' \\ \vdots \\ \vdots \\ H_{n-1'}
\end{bmatrix} = \begin{bmatrix}
3\left(\frac{d_1}{h_1} + \frac{d_n}{h_n}\right) \\
3 \left(\frac{d_1}{h_1} + \frac{d_2}{h_{2}}\right) \\ \vdots \\ 
3 \left(\frac{d_i}{h_i} + \frac{d_{i+1}}{h_{i+1}}\right) \\ 
\vdots \\ 3\left(\frac{d_{n-2}}{h_{n-2}} + \frac{d_{n-1}}{h_{n-1}}\right) \\ 

3\left(\frac{d_{n-1}}{h_{n-1}} + \frac{d_{n}}{h_{n}}\right)

\end{bmatrix}$$

La matriz de coeficientes del sistema anterior es \textit{estrictamente diagonalmente dominante} pues $\frac{2}{h_i} + \frac{2}{h_{i+1}} \geq \frac{1}{h_i} + \frac{1}{h_{i+1}}$ e igual para el primer y último elemento de la diagonal principal. Veamos que esta matriz es regular, para ello probamos el siguiente lema.

\newpage
\begin{lema}
	Toda matriz \textit{estrictamente diagonalmente dominante} es regular
\end{lema}

\begin{demostracion}
	Sea $A = (a_{ij})\in \mathcal{M}_n(\RR)$ una matriz estrictamente diagonalmente dominante, supongamos por reducción a lo absurdo que la matriz no es regular, luego existe $\bm{v} \in {\RR}^n\setminus\{\bm{0}\}$ tal que $A\bm{v} = \bm{0}$. Si ponemos $\bm{v} = (v_1, \dots, v_n)$, sea $v_i$ la componente de mayor valor absoluto, se tiene que para cada $i = 1,\dots,n$
	$$0 = \sum_{j = 1}^{n} a_{ij} v_j \iff -a_{ii} v_i = a_{i1} v_1 +\dots + a_{i, i-1}v_{i-1} + a_{i, i+1}v_{i+1} + \dots + a_{in} v_n$$
	y por tanto
	$$\abs{a_{ii} v_i} \leq \abs{a_{i1} v_1} +\dots + \abs{a_{i, i-1}v_{i-1}} + \abs{a_{i, i+1}v_{i+1}} + \dots + \abs{ a_{in} v_n}$$
	que dividiendo por $\abs{v_i}$ resulta en
	$$\abs{a_{ii}} \leq \frac{\abs{a_{i1} || v_1}}{\abs{v_i}} +\dots + \frac{\abs{a_{i, i-1} || v_{i-1}}}{\abs{v_i}} + \frac{\abs{a_{i, i+1}||v_{i+1}}}{\abs{v_i}} + \dots + \frac{\abs{ a_{in}|| v_n}}{\abs{v_i}}$$
	como $\frac{\abs{v_j}}{\abs{v_i}} \leq 1$ para cada $j=1,\dots, n$ por la elección de $v_i$ se tiene que el elemento diagonal $a_{ii}$ no excede la suma de los módulos de los elementos restantes de su columna lo cual es absurdo. Así, $A$ es regular.
\end{demostracion}

Tenemos que la matriz del sistema es regular y de dimensión $n\times n$ luego en virtud del teorema de Rouche-Frobenius el sistema tiene una única solución.
\end{document}